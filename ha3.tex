%%%%%%%%%%%%%%%%%%%%%%%%%%%%%%%%%%%%%%%%%%%%%%%%%%% Hier sollten Sie nichts verändern %%%%%%%%%%%%%%%%%%%%%%%%%%%%%%%%%%%%%%%%%%%%%%%%%%%%%%%%%%%%%%%%
\documentclass[a4paper,onecolumn,oneside,12pt,ngerman]{article}

\usepackage[ngerman]{babel} % language
\usepackage[utf8]{inputenc} % encoding
\usepackage[T1]{fontenc}  
\usepackage{amsmath,amsfonts,amssymb,amsthm} % mathematical expression, symbols, fonts, and the whole theorem environment engine
\usepackage{xcolor,graphicx} % graphics and colorings
\usepackage{tabularx,booktabs,multirow} % all about tables: booktabs gives the standard formatting, multirow allows cells to span over multiple rows.
\usepackage{scrextend} % super powerful tool that helps a bit with everything
\usepackage[margin=0pt]{subcaption} % specific form for captions
\usepackage[autostyle]{csquotes} % helper for quoting
\usepackage{listings}
\lstset{
    basicstyle=\ttfamily,
    numbers=none,
    breaklines=true,
    tabsize=4,
    showstringspaces=false,
    mathescape=true % Aktiviert LaTeX-Befehle innerhalb der Umgebung
}
\usepackage{rotating} % if you want to rotate any object

%=========== graphics, in particular tikz
\usepackage{tikz}
\usetikzlibrary{arrows,decorations.pathmorphing,decorations.pathreplacing,backgrounds,positioning,fit,matrix}
\usetikzlibrary{shapes,calc,automata}

%=========== Pseudocode
\usepackage[linesnumbered,german]{algorithm2e}

%=========== \cref is a powerfull command that automatically states the current environment (e.g. Theorem, Lemma, etc.)
\usepackage[sort&compress,nameinlink,noabbrev,capitalize]{cleveref}


%===========theorem environment
\theoremstyle{plain} % typical theorem-style (bold header, italic font)
\newtheorem{theorem}{Theorem}[section]
\newtheorem{lemma}[theorem]{Lemma}
\newtheorem{corollary}[theorem]{Korollar}
\newtheorem{observation}[theorem]{Beobachtung}
\newtheorem{proposition}[theorem]{Proposition}
\newtheorem{rrule}{Reduktionsregel}[section]

\crefname{rrule}{Reduktionsregel}{Reduktionsregeln} % how cleverref (\cref) has to handle these environments
\Crefname{rrule}{RR}{RRs} % with \Cref us can use what you defined here---somtimes useful if you want to abbreviate environment when citing (e.g. in tables or figures)

\theoremstyle{definition} % typical definition-style (bold header, normal font)
\newtheorem{definition}[theorem]{Definition}

\theoremstyle{remark} % typical theorem-style (italic header, normal font)
\newtheorem{example}{Beispiel}
\newtheorem*{remark}{Anmerkung}
\newtheorem{reduction}{Reduktion}
\theoremstyle{plain}

%===========problem definition evironment: gets three arguments (1) problem name #1 (2) input specification #2 (3) question specification #3
\newcommand{\problemdef}[3]{
  \begin{center}
    \begin{minipage}{0.95\textwidth}
      \noindent
      \textsc{#1}
      
      \vspace{2pt}
      \setlength{\tabcolsep}{3pt}
      \begin{tabularx}{\textwidth}{@{}lX@{}}
        \textbf{Eingabe:} 		& #2 \\
        \textbf{Frage:} 	& #3
      \end{tabularx}
    \end{minipage}
  \end{center}
}



%=========== Nützliche Abkürzungen
\newcommand{\NN}{\mathbb{N}} % natürliche Zahlen
\newcommand{\RR}{\mathbb{R}} % reelle Zahlen
\newcommand{\QQ}{\mathbb{Q}} % rationale Zahlen
\newcommand{\ZZ}{\mathbb{Z}} % ganze Zahlen
\newcommand{\tstep}[1][1]{\vdash^#1_M} % Turing Schritt

\date{}

%%%%%%%%%%%%%%%%%%%%%%%%%%%%%%%%%%%%%%%%%%%%%%% Ab hier können Sie das Dokument verändern %%%%%%%%%%%%%%%%%%%%%%%%%%%%%%%%%%%%%%%%%%%%%%%%%%%%%%%%%%%%%

% Hier können noch weitere \usepackage{}-Befehle eingebunden werden.
\title{3. Hausaufgabe im Modul \\ \glqq Berechenbarkeit \& Komplexität\grqq} % hier bitte die Nummer der Hausaufgabe eintragen
\author{Gruppe HA-EH-Fr-10-12-MA544-3} % Hier bitte den Namen Ihrer Gruppe (in ISIS) einfügen

\begin{document}

\selectlanguage{ngerman}

\maketitle
\newpage
Aufgabe a:\quad \textsc{\textbf{LOOP- und WHILE-Programmierung}}
\\ \\
Zuerst schauen wir uns den Basisfall $f(0,y)$ an: \\
Falls $y = 0$: \\
Da $0 = 0 \cdot k$ für alle $k \in \NN$ ist, gilt $f(0,0) = 1$\\
Falls $y > 0$:\\
Dann gibt es kein $k \in \NN$, wobei $y = 0 \cdot k$. Also $f(0,y) = 0$ \\ \\ 
\\
Als nächstes schauen wir uns die rekursive Definition an: \\ 
\[ f(x+1,y) = f (x,y) + teilbar(y, x+1) \]
\\
$f(x+1,y)$ wird berechnet, indem wir zu dem vorherigen Wert $f(x,y)$, $1$ oder $0$ addieren, abhängig davon, ob $x+1$ ein Teiler von $y$ ist.
\\
Durch das rekurisve Aufrufen werden alle Teiler von y gezählt, im Bereich von $0$ bis $x$.
\\ \\
Aus der Definition ergeben sich folgende besondere Fälle:
\\
$f(x,x)$ ist die vollständige Anzahl der Teiler von $x$, mit Ausnahme von $x=0$, da $f(x,0) = x+1$. Denn jede Zahl $x \in \NN$ wird als Teiler von $0$ gezählt, inklusive $0$ (Siehe Basisfall $y=0$). 
\\ \\ 

\noindent LOOP Pseudocode mit Eingabe $x_1, x_2$:
\begin{lstlisting}
IF $x_2 = 0$ THEN                   
    $x_0 := x_1 + 1$;    // Siehe Fall f(x,0)
    $x_1 := 0$;       // Loopaufhalter
END;
$x_3 := 1$;       // 0 wird ignoriert, siehe Basisfall y > 0
LOOP $x_1$ DO           
    $x_4 := x_2 \bmod x_3$; //  Wenn a mod b = 0, teilt b  a
    IF $x_4 = 0$ THEN 
        $x_0 := x_0 + 1$; // Ergebnis wird inkrementiert
    END;
    $x_3 := x_3 + 1$;   
END;


\end{lstlisting}

\newpage
Aufgabe b:
\\

Wir erkennen, dass $g(x,y)$ $1$  zurückgibt, falls $x$ weniger als $y$ Teiler hat (Siehe $f(x,x)$ in Aufgabe $a)$).
Es wird also $0$ zurückgegeben, wenn $x$ mindestens $y$ Teiler besitzt.
\\ \\
Der $\mu$ Operator von $g$, definiert die Funktion $f': \NN \rightarrow \NN:$

$$f'(x) = min \{ n\ |\ g(n,x) = 0\} $$
\\ \\ 
Somit gibt $f' (x)$ die kleinste Zahl $n$ zurück, die mindestens $x$ Teiler hat.
\\ \\
Besondere Fälle: \\
$f'(0) = 0$, da $f(0,0) = 1 \geq 0$ \\
$f'(1) = 0$, da $f(0,0) = 1 \geq 1$\\
\\ \\
 
Aufgabe c: \\
\\
Wir fangen die Suche nach der kleinsten Zahl bei $x_1$ an, da keine Zahl $x$ mehr als $x$ Teiler hat. Also existiert keine Zahl $x < x_1$ die mindestens $x_1$ Teiler besitzt. 
\begin{lstlisting}
$x_2 := x_1$;                   
$x_3 := P_f(x_2,x_2)$;          // $f(x_2,x_2)$ wird berechnet
$x_4 := x_1 - x_3 $;            // $x_4 = 0$ falls $f(x_2,x_2) \geq x_1$

WHILE $x_4 \neq 0$ DO
	$x_2 := x_2 + 1$;
	$x_3 := P_f(x_2,x_2)$;
	$x_4 := x_1-x_3$;
    END;
$x_0 := x_2$;
IF $x_1 = 1$ THEN $x_0 := 0$ END; // Siehe $f'(1)$ in Aufgabe b


\end{lstlisting}

\end{document}