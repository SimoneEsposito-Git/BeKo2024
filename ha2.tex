%%%%%%%%%%%%%%%%%%%%%%%%%%%%%%%%%%%%%%%%%%%%%%%%%%% Hier sollten Sie nichts verändern %%%%%%%%%%%%%%%%%%%%%%%%%%%%%%%%%%%%%%%%%%%%%%%%%%%%%%%%%%%%%%%%
\documentclass[a4paper,onecolumn,oneside,12pt,ngerman]{article}

\usepackage[ngerman]{babel} % language
\usepackage[utf8]{inputenc} % encoding
\usepackage{amsmath,amsfonts,amssymb,amsthm} % mathematical expression, symbols, fonts, and the whole theorem environment engine
\usepackage{xcolor,graphicx} % graphics and colorings
\usepackage{tabularx,booktabs,multirow} % all about tables: booktabs gives the standard formatting, multirow allows cells to span over multiple rows.
\usepackage{scrextend} % super powerful tool that helps a bit with everything
\usepackage[margin=0pt]{subcaption} % specific form for captions
\usepackage[autostyle]{csquotes} % helper for quoting

\usepackage{rotating} % if you want to rotate any object

%=========== graphics, in particular tikz
\usepackage{tikz}
\usetikzlibrary{arrows,decorations.pathmorphing,decorations.pathreplacing,backgrounds,positioning,fit,matrix}
\usetikzlibrary{shapes,calc,automata}

%=========== Pseudocode
\usepackage[linesnumbered,german]{algorithm2e}

%=========== \cref is a powerfull command that automatically states the current environment (e.g. Theorem, Lemma, etc.)
\usepackage[sort&compress,nameinlink,noabbrev,capitalize]{cleveref}


%===========theorem environment
\theoremstyle{plain} % typical theorem-style (bold header, italic font)
\newtheorem{theorem}{Theorem}[section]
\newtheorem{lemma}[theorem]{Lemma}
\newtheorem{corollary}[theorem]{Korollar}
\newtheorem{observation}[theorem]{Beobachtung}
\newtheorem{proposition}[theorem]{Proposition}
\newtheorem{rrule}{Reduktionsregel}[section]

\crefname{rrule}{Reduktionsregel}{Reduktionsregeln} % how cleverref (\cref) has to handle these environments
\Crefname{rrule}{RR}{RRs} % with \Cref us can use what you defined here---somtimes useful if you want to abbreviate environment when citing (e.g. in tables or figures)

\theoremstyle{definition} % typical definition-style (bold header, normal font)
\newtheorem{definition}[theorem]{Definition}

\theoremstyle{remark} % typical theorem-style (italic header, normal font)
\newtheorem{example}{Beispiel}
\newtheorem*{remark}{Anmerkung}
\newtheorem{reduction}{Reduktion}
\theoremstyle{plain}

%===========problem definition evironment: gets three arguments (1) problem name #1 (2) input specification #2 (3) question specification #3
\newcommand{\problemdef}[3]{
  \begin{center}
    \begin{minipage}{0.95\textwidth}
      \noindent
      \textsc{#1}
      
      \vspace{2pt}
      \setlength{\tabcolsep}{3pt}
      \begin{tabularx}{\textwidth}{@{}lX@{}}
        \textbf{Eingabe:} 		& #2 \\
        \textbf{Frage:} 	& #3
      \end{tabularx}
    \end{minipage}
  \end{center}
}



%=========== Nützliche Abkürzungen
\newcommand{\NN}{\mathbb{N}} % natürliche Zahlen
\newcommand{\RR}{\mathbb{R}} % reelle Zahlen
\newcommand{\QQ}{\mathbb{Q}} % rationale Zahlen
\newcommand{\ZZ}{\mathbb{Z}} % ganze Zahlen
\newcommand{\tstep}[1][1]{\vdash^#1_M} % Turing Schritt

\date{}

%%%%%%%%%%%%%%%%%%%%%%%%%%%%%%%%%%%%%%%%%%%%%%% Ab hier können Sie das Dokument verändern %%%%%%%%%%%%%%%%%%%%%%%%%%%%%%%%%%%%%%%%%%%%%%%%%%%%%%%%%%%%%

% Hier können noch weitere \usepackage{}-Befehle eingebunden werden.
\title{2. Hausaufgabe im Modul \\ \glqq Berechenbarkeit \& Komplexität\grqq} % hier bitte die Nummer der Hausaufgabe eintragen
\author{Gruppe HA-EH-Fr-10-12-MA544-3} % Hier bitte den Namen Ihrer Gruppe (in ISIS) einfügen

\begin{document}

\selectlanguage{ngerman}

\maketitle
\newpage
Aufgabe 1:\quad \textsc{\textbf{Turing-Maschine}}

\[ M = (Z = \{z_0, z_1, z_2, z_3, z_4\}, \Sigma = \{1\}, \Gamma = \{1,\square\},\delta, z_0, \square, E = \{z_4\}) \]

\begin{center}
	\begin{tabular}{c|ccc}
	$\delta$ & 1           & $\square$ \\ \hline
	$z_0$    & $(z_2,1,R)$ & $(z_1,1,R)$\\
	$z_1$    & $(z_3,1,L)$ & $(z_0,1,L)$\\
	$z_2$    & $(z_2,1,L)$ & $(z_0,1,L)$\\
	$z_3$    & $(z_3,\square, R)$ & $(z_4,\square,L)$\\
	$z_4$    & $\bot$ & $\bot$\\
	\end{tabular}
\end{center}

\begin{align*}
  &z_0 \square  &&\tstep  \\
  1&z_1 \square  &&\tstep  \\
  &z_0 11    &&\tstep  \\
  1&z_2 1     &&\tstep  \\
  &z_2 11    &&\tstep  \\
  &z_2 \square11   &&\tstep  \\
  &z_0 \square111  &&\tstep  \\
  1&z_1111  &&\tstep \\
  &z_3 1111  &&\tstep \\
  &z_3 111  &&\tstep \\
  &z_3 11  &&\tstep \\
  &z_3 1  &&\tstep \\
  &z_3 \square  &&\tstep \\
  &z_4 \square  
\end{align*}

Insgesamgt macht $M$ 13 Konfigurationsübergänge und hält auf $z_4$ mit leerem Band

\newpage
Aufgabe 2: \quad \textsc{\textbf{Turing-Berechenbarkeit}}

\begin{enumerate}
  \item[(a)] Die Funktion entspricht der Funktion aus Beispiel 4 der 2. Vorlesung.
  Somit ist $f$ berechenbar da es entweder konstant 1 ist, oder es existiert eine Zahl 
  $N \in \NN$ sodass
  \[f(n) = \begin{cases} 1, \quad n\leq N \\ 0, \quad \text{sonst} \end{cases}\]
  \item[(b)] Trivial:
  \[f(n) = \begin{cases}
    1, \quad n\geq 7 \\
    0, \quad \text{sonst}
  \end{cases}\]
  \item[(c)] Die Funktion ist Turing berechenbar da $\pi$ bis zu einen beliebigen Grad
  von Präzision berechnet werden kann. Daraus folgt, dass die Differenz von $\pi$ und eine 
  weitere rationale Zahl auch bis zu einem beliebigen Grad von Präzision berechenbar ist. 
\end{enumerate}

\newpage
Aufgabe 3: \quad \textsc{\textbf{LOOP- und GOTO-Programme}}

\begin{enumerate}
     \item[(a)] Nein, bei den Eingabewerten $x_1$ $\geq$ $x_2$ nicht. 
     Bei $P_1$ wird der LOOP nicht ausgeführt, somit wird auch kein Wert von $x_2$ für $x_0$ zugewiesen.
     
     \item[(b)] Nein, $P_2$ befindet sich in einer Endlosschleife bei jedem Eingabewert $x_2 > x_1$, da der Wert $x_3$ in $M_1$ nie dekrementiert wird. 

    In $P_1$ ist das dekrementieren einer Laufvariable in einem LOOP impliziert, somit befinden sich $P_1$ und $P_2$ in unterschiedlichen Endzuständen.

    \item[(c)] Ja, im Fall von $x_2 < x_1$ ist $x_3 = max(0, x_2 - x_1)$ immer $0$. Bei $P_1$ wird der LOOP nicht ausgeführt und bei $P_2$ wird direkt zu $M_2$ gesprungen und das Programm terminiert. Somit wird bei beiden Programmen der implizierte Anfangswert von $x_0 = 0$ angenommen. 

\end{enumerate}

\end{document}
