%%%%%%%%%%%%%%%%%%%%%%%%%%%%%%%%%%%%%%%%%%%%%%%%%%% Hier sollten Sie nichts verändern %%%%%%%%%%%%%%%%%%%%%%%%%%%%%%%%%%%%%%%%%%%%%%%%%%%%%%%%%%%%%%%%
\documentclass[a4paper,onecolumn,oneside,12pt,ngerman]{article}

\usepackage[ngerman]{babel} % language
\usepackage[utf8]{inputenc} % encoding
\usepackage{amsmath,amsfonts,amssymb,amsthm} % mathematical expression, symbols, fonts, and the whole theorem environment engine
\usepackage{xcolor,graphicx} % graphics and colorings
\usepackage{tabularx,booktabs,multirow} % all about tables: booktabs gives the standard formatting, multirow allows cells to span over multiple rows.
\usepackage{scrextend} % super powerful tool that helps a bit with everything
\usepackage[margin=0pt]{subcaption} % specific form for captions
\usepackage[autostyle]{csquotes} % helper for quoting

\usepackage{rotating} % if you want to rotate any object

%=========== graphics, in particular tikz
\usepackage{tikz}
\usetikzlibrary{arrows,decorations.pathmorphing,decorations.pathreplacing,backgrounds,positioning,fit,matrix}
\usetikzlibrary{shapes,calc,automata}

%=========== Pseudocode
\usepackage[linesnumbered,german]{algorithm2e}

%=========== \cref is a powerfull command that automatically states the current environment (e.g. Theorem, Lemma, etc.)
\usepackage[sort&compress,nameinlink,noabbrev,capitalize]{cleveref}


%===========theorem environment
\theoremstyle{plain} % typical theorem-style (bold header, italic font)
\newtheorem{theorem}{Theorem}[section]
\newtheorem{lemma}[theorem]{Lemma}
\newtheorem{corollary}[theorem]{Korollar}
\newtheorem{observation}[theorem]{Beobachtung}
\newtheorem{proposition}[theorem]{Proposition}
\newtheorem{rrule}{Reduktionsregel}[section]

\crefname{rrule}{Reduktionsregel}{Reduktionsregeln} % how cleverref (\cref) has to handle these environments
\Crefname{rrule}{RR}{RRs} % with \Cref us can use what you defined here---somtimes useful if you want to abbreviate environment when citing (e.g. in tables or figures)

\theoremstyle{definition} % typical definition-style (bold header, normal font)
\newtheorem{definition}[theorem]{Definition}

\theoremstyle{remark} % typical theorem-style (italic header, normal font)
\newtheorem{example}{Beispiel}
\newtheorem*{remark}{Anmerkung}
\newtheorem{reduction}{Reduktion}
\theoremstyle{plain}

%===========problem definition evironment: gets three arguments (1) problem name #1 (2) input specification #2 (3) question specification #3
\newcommand{\problemdef}[3]{
  \begin{center}
    \begin{minipage}{0.95\textwidth}
      \noindent
      \textsc{#1}
      
      \vspace{2pt}
      \setlength{\tabcolsep}{3pt}
      \begin{tabularx}{\textwidth}{@{}lX@{}}
        \textbf{Eingabe:} 		& #2 \\
        \textbf{Frage:} 	& #3
      \end{tabularx}
    \end{minipage}
  \end{center}
}



%=========== Nützliche Abkürzungen
\newcommand{\NN}{\mathbb{N}} % natürliche Zahlen
\newcommand{\RR}{\mathbb{R}} % reelle Zahlen
\newcommand{\QQ}{\mathbb{Q}} % rationale Zahlen
\newcommand{\ZZ}{\mathbb{Z}} % ganze Zahlen
\newcommand{\tstep}[1][1]{\vdash^#1_M} % Turing Schritt

\date{}

%%%%%%%%%%%%%%%%%%%%%%%%%%%%%%%%%%%%%%%%%%%%%%% Ab hier können Sie das Dokument verändern %%%%%%%%%%%%%%%%%%%%%%%%%%%%%%%%%%%%%%%%%%%%%%%%%%%%%%%%%%%%%

% Hier können noch weitere \usepackage{}-Befehle eingebunden werden.
\title{X. Hausaufgabe im Modul \\ \glqq Berechenbarkeit \& Komplexität\grqq} % hier bitte die Nummer der Hausaufgabe eintragen
\author{Gruppe XYZ} % Hier bitte den Namen Ihrer Gruppe (in ISIS) einfügen

\begin{document}

\selectlanguage{ngerman}

\maketitle
\newpage
Aufgabe 1:\quad\textsc{\textbf{Turing-Maschinen Analysieren}}
\begin{enumerate}
    \item[(a)]
    \begin{align*}
    &z_0abc  &&\tstep  \\
    a&z_0bc   &&\tstep  \\
    ab&z_1c    &&\tstep  \\
    abc&z_2\square     &&\tstep  \\
    ab&z_3c    &&\tstep  \\
    a&z_3bc   &&\tstep  \\
    &z_3abc  &&\tstep  \\
    &z_3\square bbc  &&\tstep \\
    &z_4bbc
    \end{align*}
    
    \item[(b)] 
    Wenn $M$ in $z_3$ kommt, werden danach alle 'a's mit 'b's ersetzt (von rechts nach links) bis alle buchstaben durgegangen werden, worauf $M$ im zustand $z_4$ kommt. Dann wird der Buchstabe rechts vom Lesekopf entweder ein 'b' oder ein 'c' sein.

    Aus $\delta$ folgt offensichtlich: $M$ hält falls $w\in \{a^nb^mc^k\mid n,m,k \in \NN\}$. Jedes wort was sich nicht and der Reihenfolge hält, terminiert ohne den Endzustand zu erreichen
    \item[(c)] 
    Wir betrachten das wort \enquote{aaaaaaaaa}, also $n=9$. Die Konfigurationsfolge lautet:
    \begin{align*}
    &z_0aaaaaaaaa          &&\tstep[9]  \\
    aaaaaaaaa&z_0         &&\tstep  \\
    aaaaaaaa&z_3a         &&\tstep[9]  \\
    &z_3\square bbbbbbbbb  &&\tstep  \\
    &z_4bbbbbbbbb
\end{align*}
    Da $9+1+9+1 = 20 > 18.5 = 1,5\cdot 9 + 5$ gilt die gegebene Formel nicht immer. Die richtige formel Lautet: $2n+2$
\end{enumerate}
\newpage
Aufgabe 2:\quad \textsc{\textbf{Turing-Maschinen Konstruieren}}
\[ M = (Z = \{z_0, z_1, z_2, z_3, z_4, z_e\}, \Sigma = \{a\}, \Gamma = \{a,X,\square\},\delta, z_0, \square, E = \{z_e\}) \]

\begin{center}
	\begin{tabular}{c|ccc}
	$\delta$ & a & X & $\square$ \\ \hline
	$z_0$ & $(z_1, a, R)$ & $\bot$ & $\bot$ \\
	$z_1$ & $(z_2, X, R)$ & $(z_1, X, R)$ & $(z_e, \square, N)$ \\
	$z_2$ & $(z_3, a, R)$ &$(z_2,X,R)$ & $(z_4, \square, L)$ \\
	$z_3$ & $(z_2, X, R)$ & $(z_3,X,R)$ & $\bot$ \\
	$z_4$ & $(z_4, a, L)$ & $(z_4, X, L)$ & $(z_0, \square, R)$ \\
	\end{tabular}
\end{center}

Erklärung:

\begin{itemize}
	\item Das symbol $X$ wird von der Maschine immer ignoriert (siehe $\delta$ für $X$).
	\item $z_0$: verifiziert, dass das Word mit 'a' anfängt, 
	sonst hält die Maschine ohne den Endzustand zu erreichen.
	\item $z_1$: falls das Wort nur einen 'a' enthält, wird es akzeptiert. 
	Sonst wird das erste 'a' durch 'X' markiert und die Maschine geht in $z_2$.
	\item $z_2$ und $z_3$: halbieren die 'a's in $w$ indem jedes zweite 'a' von $z_3$
	mit 'X' markiert wird. Da das word mit 'a' anfängt, wenn die Maschine im zustand $z_3$
	am Ende des Wortes ankommt (z.B: $aXaXaz_3$), heißt es, dass das Wort eine ungerade 
	Anzahl von 'a's enthält. Somit hält die Maschine ohne den Endzustand zu erreichen.
	(außer wenn das Wort nur ein 'a' enthält, dann wird es akzeptiert, siehe $z_1$).
	\item $z_4$: Falls das Wort eine gerade Anzahl von 'a's enthält, kommt die Maschine
	in $z_4$ an, was den Lesekopf wieder auf das erste 'a' setzt und die Maschine in $z_0$ geht.
	\item Der Prozess der Halbierung wird wiederholt, bis das Wort nur ein 'a' enthält.
	Falls das Wort an einer Iteration eine ungerade Anzahl von 'a's enthält, hält die Maschine ohne den Endzustand zu erreichen.
\end{itemize}
Dies funktioniert, da:
\[x = 2^n \Leftrightarrow \frac{x}{2} = 2^{n-1} \Leftrightarrow \frac{x}{2^i} = 2^{n-i} \Leftrightarrow \frac{x}{2^n} = 2^{n-n} = 1\]

\end{document}
