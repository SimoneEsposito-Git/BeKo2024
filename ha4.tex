%%%%%%%%%%%%%%%%%%%%%%%%%%%%%%%%%%%%%%%%%%%%%%%%%%% Hier sollten Sie nichts verändern %%%%%%%%%%%%%%%%%%%%%%%%%%%%%%%%%%%%%%%%%%%%%%%%%%%%%%%%%%%%%%%%
\documentclass[a4paper,onecolumn,oneside,12pt,ngerman]{article}

\usepackage[ngerman]{babel} % language
\usepackage[utf8]{inputenc} % encoding
\usepackage[T1]{fontenc}  
\usepackage{amsmath,amsfonts,amssymb,amsthm} % mathematical expression, symbols, fonts, and the whole theorem environment engine
\usepackage{xcolor,graphicx} % graphics and colorings
\usepackage{tabularx,booktabs,multirow} % all about tables: booktabs gives the standard formatting, multirow allows cells to span over multiple rows.
\usepackage{scrextend} % super powerful tool that helps a bit with everything
\usepackage[margin=0pt]{subcaption} % specific form for captions
\usepackage[autostyle]{csquotes} % helper for quoting

\usepackage{rotating} % if you want to rotate any object

%=========== graphics, in particular tikz
\usepackage{tikz}
\usetikzlibrary{arrows,decorations.pathmorphing,decorations.pathreplacing,backgrounds,positioning,fit,matrix}
\usetikzlibrary{shapes,calc,automata}

%=========== Pseudocode
\usepackage[linesnumbered,german]{algorithm2e}

%=========== \cref is a powerfull command that automatically states the current environment (e.g. Theorem, Lemma, etc.)
\usepackage[sort&compress,nameinlink,noabbrev,capitalize]{cleveref}


%===========theorem environment
\theoremstyle{plain} % typical theorem-style (bold header, italic font)
\newtheorem{theorem}{Theorem}[section]
\newtheorem{lemma}[theorem]{Lemma}
\newtheorem{corollary}[theorem]{Korollar}
\newtheorem{observation}[theorem]{Beobachtung}
\newtheorem{proposition}[theorem]{Proposition}
\newtheorem{rrule}{Reduktionsregel}[section]

\crefname{rrule}{Reduktionsregel}{Reduktionsregeln} % how cleverref (\cref) has to handle these environments
\Crefname{rrule}{RR}{RRs} % with \Cref us can use what you defined here---somtimes useful if you want to abbreviate environment when citing (e.g. in tables or figures)

\theoremstyle{definition} % typical definition-style (bold header, normal font)
\newtheorem{definition}[theorem]{Definition}

\theoremstyle{remark} % typical theorem-style (italic header, normal font)
\newtheorem{example}{Beispiel}
\newtheorem*{remark}{Anmerkung}
\newtheorem{reduction}{Reduktion}
\theoremstyle{plain}

%===========problem definition evironment: gets three arguments (1) problem name #1 (2) input specification #2 (3) question specification #3
\newcommand{\problemdef}[3]{
  \begin{center}
    \begin{minipage}{0.95\textwidth}
      \noindent
      \textsc{#1}
      
      \vspace{2pt}
      \setlength{\tabcolsep}{3pt}
      \begin{tabularx}{\textwidth}{@{}lX@{}}
        \textbf{Eingabe:} 		& #2 \\
        \textbf{Frage:} 	& #3
      \end{tabularx}
    \end{minipage}
  \end{center}
}



%=========== Nützliche Abkürzungen
\newcommand{\NN}{\mathbb{N}} % natürliche Zahlen
\newcommand{\RR}{\mathbb{R}} % reelle Zahlen
\newcommand{\QQ}{\mathbb{Q}} % rationale Zahlen
\newcommand{\ZZ}{\mathbb{Z}} % ganze Zahlen
\newcommand{\tstep}[1][1]{\vdash^#1_M} % Turing Schritt

\date{}

%%%%%%%%%%%%%%%%%%%%%%%%%%%%%%%%%%%%%%%%%%%%%%% Ab hier können Sie das Dokument verändern %%%%%%%%%%%%%%%%%%%%%%%%%%%%%%%%%%%%%%%%%%%%%%%%%%%%%%%%%%%%%

% Hier können noch weitere \usepackage{}-Befehle eingebunden werden.
\title{4. Hausaufgabe im Modul \\ \glqq Berechenbarkeit \& Komplexität\grqq} % hier bitte die Nummer der Hausaufgabe eintragen
\author{Gruppe HA-EH-Fr-10-12-MA544-3} % Hier bitte den Namen Ihrer Gruppe (in ISIS) einfügen

\begin{document}

\selectlanguage{ngerman}

\maketitle
\newpage
Aufgabe 1:\quad \textsc{\textbf{Entscheidbarkeit}}
\vspace{20pt}\\
Wir Zeigen dass die Sprachen $L_1, \ldots, L_n$ semi-entscheidbar sind analog zu der VL7.\\
Da $L_1$ bis $L_n$ semi-entscheidbar sind, heißt es dass jede Sprache $\chi'_{L_i}$ 
berechenbar durch WHILE-Programme mit einer WHILE-Schleife:

\begin{algorithm}[H]
    $x_i \leftarrow 1$\;
    \While{$x_i \neq 0$}{
        $P_{L_i}$\;
    }
    $x_0 \leftarrow 1$\;
\end{algorithm}
\hfill\\
Dann können wir ein Programm Konstruiren das die Sprache $L_i$ entscheidet indem wir
nacheinander $P_{L_j}$ für jedes $j \in \{1, \ldots, n\}$ ausführen.\\
Da jedes $L_j$ semi-entscheidbar ist und $\Cup^n_{i=1} L_i = \Sigma^\ast$ ist,
wird das Programm bei einer Eingabe $w\in\Sigma^\ast$ eventuell für ein $j$ terminieren. Falls $j = i$, dann geben wir 1 aus, sonst 0.

\begin{algorithm}[H]
    $x_1 \leftarrow 1$; $x_2 \leftarrow 1$; ...; $x_{n+1} \leftarrow 1$\;
    \While{$x_1 \neq 0$ und $x_2 \neq 0$ und ... und $x_{n+1} \neq 0$}{
        $P_{L_0}$; $P_{L_1}$; ...; $P_{L_n}$\;
    }
    \eIf{$x_{j+1}=1$}{$x_0=1$}{$x_0=0$}
\end{algorithm}
\newpage
Aufgabe 2:\quad \textsc{\textbf{Abgeschlossenheit Semi-Entscheidbarkeit}}
\vspace{20pt}
\begin{enumerate}
    \item[(a)] Gegenbeispiel: Sei 
    \[U := \{w\in\{0,1\}^\ast\} = \{0,1\}^\ast\] 
    Wir erkennen dass, $U$ entscheidbar (und somit auch semi-entscheidbar)
    ist, da wir 1 ausgeben für jede Eingabe $w\in\{0,1\}^\ast$ und 0 für jede Eingabe $w\notin\{0,1\}^\ast$.
    Aus der VL6 kennen wir außerdem das spezielle Halteproblem 
    \[K := \{w \in \{0,1\}^\ast | M_w \text{ hält auf Eingabe }w\}\]
    Wir erkennen, dass $K$ semi-entscheidbar ist, da wir 1 ausgeben wenn $M_w$ hält, sonst wird $M_w$ nicht halten.
    Aus der VL6 wissen wir auch, dass $K$ unentscheidbar ist.\\
    Offenbar gilt $K \subseteq U$, da $K$ nur aus Wörtern besteht, die aus 0 und 1 bestehen.\\
    Wenn $U\setminus K$ semi-entscheidbar wäre, dann wäre $K$ entscheidbar (VL7), was ein Widerspruch ist.
    \item[(b)] da $L_1$ und $L_2$ semi-entscheidbar sind, gibt es zwei Turingmaschinen $M_1$ und $M_2$
    die Halten wenn $w\in L_1$ bzw. $w\in L_2$.\\
    Wir können eine Turingmaschine $M$ konstruieren, die $L = \{w_1w_2 | w_1 \in L_1, w_2 \in L_2\}$ auch semi-entscheidet indem wir
    ein Wort $w$ aufteilen in wörtern $w_{1,i} =$ Präfix der länge $i$ und $w_{2,i}=$ Suffix der länge $|w|-i$ für $i=0,1,...,|w|$.
    Wir führen dann ein Schritt von $M_1$ auf $w_{1,i}$ und ein Schritt von $M_2$ auf $w_{2,i}$ aus für jedes $i$.
    Falls $w\in L$ wird es nach definition von $L$ wird es ein $i$ geben sodass $w_{1,i} \in L_1$ und $w_{2,i} \in L_2$. 
    So geben wir 1 aus, falls $M_1$ und $M_2$ beide halten, sonst wird das Programm nicht halten. Somit ist $L$ semi-entscheidbar.
\end{enumerate}
\newpage
Aufgabe 3:\quad \textsc{\textbf{Nicht (co-)semi-entscheidbare Sprachen}}
\vspace{20pt}\\
Seien $S_i \subseteq \{0,1\}^\ast, \ i\in\NN$ alle semi- und co-semi-entscheidbare Sprachen.\\
Wir können anhand von Diagonalisierung wie folgt eine Sprache $D$ konstruieren:
\[D = \{w_i \in \{0,1\}^\ast| w_i \not\in S_i\}\quad i\in\NN\]
Da kein wort $w \in D$ in $S_i$ enthalten ist, existiert keine funktion die $\chi'_D$ berechnet.\\
Somit ist $D$ weder semi- noch co-semi-entscheidbar.
\end{document}
